\documentclass[12pt]{article}
\begin{document}

\section{Buildings and population}

The main effect of buildings is to give a bonus to population. This
happens in one of two ways. The base weight of a province is given by
$M+P$, manpower plus production, both integrated over time, where
\begin{eqnarray*}
M &=& 0.125*m*b \\
P &=& b*p \\
b &=& 0.05*t + \min(2.0, 0.99+(c/101000))
\end{eqnarray*}
where $p$ is the price of the trade good in the province, $m$ is
the `manpower' number listed in the save, $t$ is the basetax, and $c$
is the city size. Buildings that affect manpower or production thus
give additional weight just by giving bonuses to these
numbers. Buildings that don't, instead give a percentage bonus
proportional to $\sqrt{0.001*cost}$. This quantity is multiplied by
the time the building the time the building it has existed, divided by
the time it \emph{could} have existed; so a building invented in 1399
and built in 1599 gives half its maximum bonus to a conversion in
1799. 

The integration over time also strives to take historical slider
positions, decisions, and triggered modifiers into account, but does
not do so perfectly. Slider changes, for example, are not stored in
the save, only the initial and final positions, so I use a heuristic
that most of the changes occur early in the history, and
interpolate. This also means that if you start, say, at 0, are at 5
for much of your history, and end at 3, the converter will think you
were between 0 and 3 for the whole game. Can't be helped! 

There are some other minor modifiers to population:
\begin{itemize}
\item Capitalists only go where there is a stock exchange. 
\item Craftsmen and clerks only go where there are factories, and
  vice-versa. 
\item Nations with the abolish slavery decision get no slaves. 
\item Some trade goods attract or repel some pop types, as shown in
  config.txt. For example, tobacco, slaves, sugar and cotton all
  attract slaves. 
\item Slider positions and national ideas have a small effect in
  attracting some pop types; again this is shown in config.txt. For
  example, Free Subjects is attractive to labourers and farmers, while
  Plutocracy attracts capitalists. 
\item Occupations and low religious tolerance reduce population
  weight, integrated over time. A province which was occupied by
  someone other than its owner for the whole of EU3 would only get
  10\% of the population it would otherwise have. A province which was
  consistently at -3 tolerance would get 50\% of its population. 
\end{itemize}

Additionally, some buildings affect literacy. In particular, each
building has a literacy weight (which defaults to zero) in
buildings.txt. The literacy weight of a province is calculated thus:
For each province in the world, add up the literacy weight of its
buildings and divide by the distance (plus 1) between the two
provinces. Integrate this quantity over time. Then redistribute the
vanilla literacy according to this literacy weight. 

\section{Sliders}

Sliders affect population as outlined in the previous section; trade
power, which affects technology as discussed below; literacy, and
politics. Each pop type has a `literacy' section in config.txt,
showing the effect of sliders. Notice that these are percentage
bonuses to the literacy weight, not direct modifiers to the literacy;
note also that they are not integrated over time. 

In addition to the effect on parties, sliders can affect reforms. 
Government forms which require some voting to exist get the landed
voting reform; others get nothing. For the other political reform
types (social reforms are all at the lowest possible upon conversion),
nations earn points as shown in config.txt. For example, consider the
slavery entry:
\begin{verbatim}
slavery={
  base=no_slavery
  provinces={
    cotton = 2
    slaves = 2
    tobacco = 1.5
    sugar = 2
  }
  modifier = {
    mod = the_abolish_slavery_act
    mult = 0
  }
  slider = {
    which = serfdom_freesubjects
    value = -0.03
    max = 0 
  }
  threshold = {
    value = 0.15
    rhs = yes_slavery 
  }
}
\end{verbatim}
This says that the base position is ``No Slavery''. The 'threshold'
entry says that if the nation has more than 0.15 points, the value
instead becomes ``Slavery Allowed''; for other reforms there can be
more than one threshold. The slider entry indicates that each click
towards serfdom gives 0.03 points, but clicks towards freedom do not
give additional points (``max $=$ 0''). The provinces entry
says that provinces having the goods indicated give an amount of
points equal to their percentage of the weighted sum of total national
basetax, where the weights are equal to the numbers given, or 1. (That
is to say, consider a nation with two provinces, one with slaves and
basetax 2, the other with gold and basetax 3. The total weight is
$(2*2 + 1*3)=7$, and the slave province will therefore give
$4/7\approx 0.57$ points towards the slavery reform. Finally, the
'mod' field indicates that if the nation has the national modifier in
question, the points value is multiplied by 'mult' - in other words,
nations which have abolished slavery do not get the ``Slavery
Allowed'' political reform, however far they are towards serfdom or
however many cotton, slave, tobacco and sugar provinces they have. 

Sliders also affect national values as shown in natvalues.txt. The
numbers here are to be read as weights which each click along the
slider gives to the value in question; the highest weight wins. 

\section{National ideas}

These affect literacy, political parties, reforms, and national values
basically in the same way as sliders. 

\section{Parties}

Each nation gets a ``historical'' packet of parties, that is, one
belonging to a nation in vanilla; for example, you may get the
Prussian parties, the English parties, and so on. You get the parties
of that historical nation which you most resemble, unless someone else
resembles it even more; in that case you get the
second-most-resemblant tag's parties, and so on. ``Resemblance'' is
based on sliders, decisions, ideas, and kustom; the resemblance
objects can be found in config.txt. For example:


\begin{verbatim}
SWE = {
  ideas = {
    church_attendance_duty = 1
    humanist_tolerance = 1
    ecumenism = 1
  }
  sliders = {
    quality_quantity = -0.25
  }
  modifiers = {
    swedish_steel = 1
    indelningsverket = 1
    sunday_schools = 1
    finnish_cavalry = 1
  }
}
\end{verbatim}

So, a nation with the three listed ideas, the four listed modifiers,
and full Quality (ie, -5 Quality) would get 8.25 resemblance points
for Sweden, plus whatever it added by kustom. Here's an example of how
it works: Let there be three EU3 nations being converted, FIC, UTO,
and ATL; and three Vicky nations listed in config.txt for them to
resemble, say SWE, RUS, and FRA. Further, let's have:

\begin{verbatim}
FIC resembles 
  SWE = 11
  RUS = 12
  FRA = 4
UTO resembles
  SWE = 12
  RUS = 20
  FRA = 1
ATL resembles
  SWE = 9
  RUS = 2
  FRA = 7
\end{verbatim}

The highest resemblance here is UTO's 20 for RUS, so the first
assignment is UTO to RUS. The next highest resemblance is FIC for RUS,
with 12 - but RUS is already taken, so this is ignored. Next is FIC's
resemblance for SWE, with 11 points, so this is assigned. Finally ATL
gets FRA as the only remaining option. If ATL had used its 1000 kustom
points to increase its resemblance to SWE, with a \verb|SWE=1.0| entry in
its kustom, that resemblance would have been 19 (ie, 0.01 resemblance
points per custom point - this is set in config.txt, by the field
\verb|partyPointsPerCustom|)
and it would have beaten out FIC for the Swedish parties. On the other hand, had it
instead put \verb|RUS=1.0|, its weight would only have been 12, and the
kustom points would have been lost to no effect. 

\section{Technology}

Affects starting research points and civilised status. Nations which
have at least four of the five tech areas within one level of
up-to-date will convert civilised. Additionally, each tech area gives
$30000*(N/80)^2$ research points where $N$ is the EU3 tech
level. Finally, the most advanced tech level decides which research
institution the nation gets. 

In addition, trade power, which is calculated as a nation's share of
the total trade income of the world (approximately - the calculation
doesn't catch every possible modifier to trade income), with a
time-integrated bonus from customs houses, is applied as a percentage
bonus to RPs. 

\section{Customisation}

Parties and initial techs are somewhat customisable; in addition,
nations may spend customisation points on increasing their starting
RPs or cash. By default, every nation uses all its custom points
(which are acquired through having buildings that have a
\verb|customPoints| entry in \verb|buildings.txt|) on RPs, and uses
those initial RPs to buy techs in the order listed in the DUMMY field
of Custom.txt. However, each nation may supply a custom object, thus:
\begin{verbatim}
FIC = {
  # FIC has 100 custom points
  rp = 0.75        # Use 75 of them to get 750 extra RPs
  money = 0.15     # Use 15 to get 1500 extra cash
  USA = 0.1        # Use 10 to increase weight for the USA set of parties by 0.1
  research = {
    tech = late_enlightenment_philosophy  # First tech to buy
    tech = freedom_of_trade               # Next tech
    tech = private_banks
    tech = tanks                          # This will not in fact happen - date after 1840. 
    tech = water_wheel_power
    tech = practical_steam_engine
    tech = high_n_low_pressure_steam_engines
    tech = experimental_railroad
    tech = flintlock_rifles
  }
}
\end{verbatim}
The entries \texttt{rp}, \texttt{money}, and \texttt{NNN} (ie, a Vicky
tag) indicate what fraction of custom points the nation would like to
use for RPs, money, or to increase its resemblance to the Vicky tag
for purposes of getting parties; the rates of conversion are all
configurable. 



\section{Miscellanea} 

\begin{itemize}
\item Armies and navies convert proportionally, that is, if you have
  one-tenth of the EU3 infantry you will get (approximately) one-tenth
  of the vanilla Vicky infantry. Note that this only applies to quite
  advanced EU3 units; low-tech armies convert to irregulars (still on
  the proportional system) and low-tech ships don't convert at all. 
\item Infamy converts one-for-one. 
\item Inflation converts to debt.
\item High-level forts and ports convert to Vicky forts and naval
  bases.  
\item Government types convert as shown in config.txt. 
\item Army and navy tradition convert to leadership. 
\end{itemize}

\end{document} 
