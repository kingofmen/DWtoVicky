\documentclass[12pt]{article}
\begin{document}

\section{Buildings and population}

The main effect of buildings is to give a bonus to population. This
happens in one of two ways. The base weight of a province is given by
$M+P$, manpower plus production, both integrated over time, where
\begin{eqnarray*}
M &=& 0.125*m*b \\
P &=& b*p \\
b &=& 0.05*t + \min(2.0, 0.99+(c/101000))
\end{eqnarray*}
where $p$ is the price of the trade good in the province, $m$ is
the `manpower' number listed in the save, $t$ is the basetax, and $c$
is the city size. Buildings that affect manpower or production thus
give additional weight just by giving bonuses to these
numbers. Buildings that don't, instead give a percentage bonus
proportional to $\sqrt{0.001*cost}$. This quantity is multiplied by
the time the building the time the building it has existed, divided by
the time it \emph{could} have existed; so a building invented in 1399
and built in 1599 gives half its maximum bonus to a conversion in
1799. 

The integration over time also strives to take historical slider
positions, decisions, and triggered modifiers into account, but does
not do so perfectly. Slider changes, for example, are not stored in
the save, only the initial and final positions, so I use a heuristic
that most of the changes occur early in the history, and
interpolate. This also means that if you start, say, at 0, are at 5
for much of your history, and end at 3, the converter will think you
were between 0 and 3 for the whole game. Can't be helped! 

There are some other minor modifiers to population:
\begin{itemize}
\item Capitalists only go where there is a stock exchange. 
\item Craftsmen and clerks only go where there are factories, and
  vice-versa. 
\item Nations with the abolish slavery decision get no slaves. 
\item Some trade goods attract or repel some pop types, as shown in
  config.txt. For example, tobacco, slaves, sugar and cotton all
  attract slaves. 
\item Slider positions and national ideas have a small effect in
  attracting some pop types; again this is shown in config.txt. For
  example, Free Subjects is attractive to labourers and farmers, while
  Plutocracy attracts capitalists. 
\item Occupations and low religious tolerance reduce population
  weight, integrated over time. A province which was occupied by
  someone other than its owner for the whole of EU3 would only get
  10\% of the population it would otherwise have. A province which was
  consistently at -3 tolerance would get 50\% of its population. 
\end{itemize}

Additionally, some buildings affect literacy. In particular, each
building has a literacy weight (which defaults to zero) in
buildings.txt. The literacy weight of a province is calculated thus:
For each province in the world, add up the literacy weight of its
buildings and divide by the distance (plus 1) between the two
provinces. Integrate this quantity over time. Then redistribute the
vanilla literacy according to this literacy weight. 

\section{Sliders}

Sliders affect population as outlined in the previous section; trade
power, which affects technology as discussed below; literacy, and
politics. Each pop type has a `literacy' section in config.txt,
showing the effect of sliders. Notice that these are percentage
bonuses to the literacy weight, not direct modifiers to the literacy;
note also that they are not integrated over time. 

Sliders also affect the customisation of parties. Each party type in
the ideologies field of config.txt has an entry for each attitude, and
each such entry (for example, for jingoism) has a sliders section. The
numbers are to be read as percentage modifiers to the base weight
that the party in question will have that attitude. Thus, for example,
consider the entry for conservatives:

\begin{verbatim}
conservative = {
  start_date = 1830.1.1 
  jingoism = {
    ideas = {
      national_conscripts = 1.25  
      battlefield_commissions = 1.25
      military_drill = 1.25
      deus_vult = 1.1
    }
    sliders = {
      land_naval = -0.03 
    }
  }
}
\end{verbatim}

There is some base weight for conservatives to be jingoist, which is
equal to the fraction of vanilla conservative parties that are
jingoist. It is modified by slider positions; in this case, each click
towards naval decreases the base weight by 3\%. 
The \emph{probability} that they are jingoist is equal to this modified
weight, divided by the sum of the modified weights for jingoism, pacifism,
anti-military, and pro-military. 

In addition to the effect on parties, sliders can affect reforms. 
Government forms which require some voting to exist get the landed
voting reform; others get nothing. For the other political reform
types (social reforms are all at the lowest possible upon conversion),
nations earn points as shown in config.txt. For example, consider the
slavery entry:
\begin{verbatim}
slavery={
  base=no_slavery
  provinces={
    cotton = 2
    slaves = 2
    tobacco = 1.5
    sugar = 2
  }
  modifier = {
    mod = the_abolish_slavery_act
    mult = 0
  }
  slider = {
    which = serfdom_freesubjects
    value = -0.03
    max = 0 
  }
  threshold = {
    value = 0.15
    rhs = yes_slavery 
  }
}
\end{verbatim}
This says that the base position is ``No Slavery''. The 'threshold'
entry says that if the nation has more than 0.15 points, the value
instead becomes ``Slavery Allowed''; for other reforms there can be
more than one threshold. The slider entry indicates that each click
towards serfdom gives 0.03 points, but clicks towards freedom do not
give additional points (``max $=$ 0''). The provinces entry
says that provinces having the goods indicated give an amount of
points equal to their percentage of the weighted sum of total national
basetax, where the weights are equal to the numbers given, or 1. (That
is to say, consider a nation with two provinces, one with slaves and
basetax 2, the other with gold and basetax 3. The total weight is
$(2*2 + 1*3)=7$, and the slave province will therefore give
$4/7\approx 0.57$ points towards the slavery reform. Finally, the
'mod' field indicates that if the nation has the national modifier in
question, the points value is multiplied by 'mult' - in other words,
nations which have abolished slavery do not get the ``Slavery
Allowed'' political reform, however far they are towards serfdom or
however many cotton, slave, tobacco and sugar provinces they have. 


Sliders also affect national values as shown in natvalues.txt. The
numbers here are to be read as weights which each click along the
slider gives to the value in question; the highest weight wins. 

\section{National ideas}

These affect literacy, political parties, reforms, and national values
basically in the same way as sliders. 

\section{Technology}

Affects starting research points and civilised status. Nations which
have at least four of the five tech areas within one level of
up-to-date will convert civilised. Additionally, each tech area gives
$30000*(N/80)^2$ research points where $N$ is the EU3 tech
level. Finally, the most advanced tech level decides which research
institution the nation gets. 

In addition, trade power, which is calculated as a nation's share of
the total trade income of the world (approximately - the calculation
doesn't catch every possible modifier to trade income), with a
time-integrated bonus from customs houses, is applied as a percentage
bonus to RPs. 

\section{Customisation}

Parties and national stockpiles may be customised. Additionally, for
AHD, the initial techs are customisable. 

Initially parties are generated randomly as outlined in the section on
sliders. However, they may then be customised. Each country gets a
number of customisation points, calculated as follows. Government
buildings have a 'moddingPoints' entry; the total modding points of
a nation, divided by its provinces, is its modding fraction; 
customisation points are distributed in proportion to the modding
fractions. Example: England has 1 temple giving a single modding
point, and two provinces; its modding fraction is 0.5. Germany has two
temples and three provinces, giving a modding fraction of
0.66. England will get $0.5/(0.5+0.66)=43\%$ of the total
customisation points. 

The customisation points are calculated thus: Every time a political
party is randomly generated, there was some probability that it would
end up as it did, based on the frequencies of the vanilla parties. For
example, suppose that 80\% of vanilla conservatives are jingoist, and
the rest are pro-military; then a pro-military conservative party has
a 20\% probability, multiplied by the similar probabilities from the
other fields. The negative log of this probability is that party's contribution
to the customisation pool. 

Now, players may customise their nations, in this format:
\begin{verbatim}
KHA = {
  liberal = {
    area = trade_policy
    position = free_trade 
  }
  resource = {
    which = iron
    amount = 2
  } 
  research = {
    tech = late_enlightenment_philosophy
    tech = freedom_of_trade
    tech = private_banks
    tech = water_wheel_power
  }
}
\end{verbatim}
This says that the Khanate wants liberals who are in favour of Free
Trade. (If it wanted to specify liberal positions on other points, it
should make another 'liberals' entry.) Each country will attempt to
buy the listed customisations in the order given; if it can't afford
one, that entry is skipped. The cost of a customisation is equal to
the negative log of its probability ratio, times the number of
previously successful bids plus one. For example, suppose I want
pro-military conservatives, probability 20\%, but I got
randomly-generated jingoists, probability 80\%. (If they were randomly
generated as pro-military, obviously, the customisation is skipped.)
Then the cost of the customisation is $-\log(0.2/0.8)=1.38$, times
however many previous bids I had. The minimum cost is the number of
prior successful bids, plus one; customisations that ``go with the
flow'', ie in the direction of higher probability, will generally have
the minimum cost. On the other hand, laissez-faire communists are
going to be highly expensive. 

An `unless' clause may be added to a bid, in which case the bid will
only be used if the party in question has a position other than the
one indicated by the unless. 

The 'resource' bid says that the nation would like its stockpile
increased by 20 iron. Only resources which exist in some nation's
stockpile at game start can be bid on in this way. The cost is equal
to the negative logarithm of that resource's fraction of the world
stockpile, times the number of units, times the number of previous
successful bids plus one. 

The research entry is not a bid, but just a list of the order in which
the nation would like to use its research points to buy initial
techs. 

\section{Miscellanea} 

\begin{itemize}
\item Armies and navies convert proportionally, that is, if you have
  one-tenth of the EU3 infantry you will get (approximately) one-tenth
  of the vanilla Vicky infantry. Note that this only applies to quite
  advanced EU3 units; low-tech armies convert to irregulars (still on
  the proportional system) and low-tech ships don't convert at all. 
\item Infamy converts one-for-one. 
\item Inflation converts to debt.
\item High-level forts and ports convert to Vicky forts and naval
  bases.  
\item Government types convert as shown in config.txt. 
\item Army and navy tradition convert to leadership. 
\end{itemize}

\end{document} 
